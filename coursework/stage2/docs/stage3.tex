\documentclass[12pt,onecolumn]{article}
\usepackage[utf8]{inputenc} % UTF8 input encoding
\usepackage[T2A]{fontenc}   % T2A font encoding for Cyrillic script
\usepackage[russian]{babel} % Russian language support
\usepackage{listings}
\usepackage{float}
\usepackage{mathtools}
\usepackage{longtable}
\everymath{\displaystyle}
\usepackage[utf8]{inputenc}
\usepackage{listings} 
\usepackage[usenames]{color}
\usepackage{geometry}
\usepackage{verbatim}
\geometry{
  a4paper,
  top=25mm, 
  right=15mm, 
  bottom=25mm, 
  left=15mm
}

\begin{document}
\setcounter{tocdepth}{4}
\begin{center}
    Федеральное государственное автономное образовательное учреждение высшего образования "Национальный Исследовательский Университет ИТМО"\\ 
    Мегафакультет Компьютерных Технологий и Управления\\
    Факультет Программной Инженерии и Компьютерной Техники \\
    \includegraphics[scale=0.3]{image/itmo.jpg} % нужно закинуть картинку логтипа в папку с отчетом
\end{center}
\vspace{1cm}


\begin{center}
    \textbf{Курсовая работа этап 3}\\
    по дисциплине\\
    \textbf{'Информационные системы и базы данных'}\\
\end{center}

\vspace{2cm}

\begin{flushright}
  Выполнили Студенты  группы P33102\\
  \textbf{Лапин Алексей Александрович}\\
  \textbf{Юнусов Роман Ильдарович}\\
  Преподаватель: \\
  \textbf{Сагайдак Алина Алексеевна}\\
\end{flushright}

\vspace{6cm}
\begin{center}
    г. Санкт-Петербург\\
    2023г.
\end{center}

\newpage
\tableofcontents
\newpage
\section{Текст задания.}
Реализовать даталогическую модель в реляционной СУБД PostgreSQL: 
\begin{itemize}
  \item Создать необходимые объекты базы данных.
  \item Заполнить созданные таблицы тестовыми данными.
  \item Сделать скрипты для: \begin{itemize}
    \item создания/удаления объектов базы данных;
    \item заполнения/удаления созданных таблиц.
  \end{itemize}
  \item Обеспечить целостность данных при помощи средств языка DDL.
  \item Добавить в базу данных триггеры для обеспечения комплексных ограничений
  целостности.
  \item Реализовать функции и процедуры на основе описания бизнес-процессов (из этапа
  №1).
  \item Произвести анализ использования созданной базы данных: \begin{itemize}
    \item выявить наиболее часто используемые запросы к объектам базы данных;
    \item результаты представить в виде текстового описания. 
  \end{itemize}
  \item Создать индексы и доказать, что они полезны для вашей базы данных: \begin{itemize}
    \item доказательство должно быть приведено в виде текстового описания.
  \end{itemize}
\end{itemize}
\section{Описание предметной области.}
Магазин покемонов.\\
Существуют тренеры и покемоны. В нашем магазине продаются покемоны и камни эволюции. Наш магазин умеет предлагать к покупке покемонов соответствующие уровню и стилю игры тренера. Некоторые из покемонов могу эволюционировать в других покемонов при применении подходящего камня эволюции. 
\section{Даталогическая модель.}
\includegraphics[width=\textwidth]{image/datalogical-model.png}
\section{Создать необходимые объекты базы данных.}
\lstdefinestyle{sql}{language=SQL, 
  basicstyle=\small\ttfamily,
  commentstyle=\color{cyan},
  stringstyle=\color{magenta}\ttfamily,
  keywordstyle=\color{blue},
  numbers=left,
  numberstyle=\scriptsize,
  numbersep=5pt,
  frame=single,
  breaklines=true,
  breakatwhitespace=true,
  showstringspaces=false,
  tabsize=4,
  inputencoding=utf8,
  extendedchars=true,
  literate={а}{{\selectfont\char224}}1
          {б}{{\selectfont\char225}}1
          {в}{{\selectfont\char226}}1
          {г}{{\selectfont\char227}}1
          {д}{{\selectfont\char228}}1
          {е}{{\selectfont\char229}}1
          {ё}{{\"e}}1
          {ж}{{\selectfont\char230}}1
          {з}{{\selectfont\char231}}1
          {и}{{\selectfont\char232}}1
          {й}{{\selectfont\char233}}1
          {к}{{\selectfont\char234}}1
          {л}{{\selectfont\char235}}1
          {м}{{\selectfont\char236}}1
          {н}{{\selectfont\char237}}1
          {о}{{\selectfont\char238}}1
          {п}{{\selectfont\char239}}1
          {р}{{\selectfont\char240}}1
          {с}{{\selectfont\char241}}1
          {т}{{\selectfont\char242}}1
          {у}{{\selectfont\char243}}1
          {ф}{{\selectfont\char244}}1
          {х}{{\selectfont\char245}}1
          {ц}{{\selectfont\char246}}1
          {ч}{{\selectfont\char247}}1
          {ш}{{\selectfont\char248}}1
          {щ}{{\selectfont\char249}}1
          {ъ}{{\selectfont\char250}}1
          {ы}{{\selectfont\char251}}1
          {ь}{{\selectfont\char252}}1
          {э}{{\selectfont\char253}}1
          {ю}{{\selectfont\char254}}1
          {я}{{\selectfont\char255}}1
          {А}{{\selectfont\char192}}1
          {Б}{{\selectfont\char193}}1
          {В}{{\selectfont\char194}}1
          {Г}{{\selectfont\char195}}1
          {Д}{{\selectfont\char196}}1
          {Е}{{\selectfont\char197}}1
          {Ё}{{\"E}}1
          {Ж}{{\selectfont\char198}}1
          {З}{{\selectfont\char199}}1
          {И}{{\selectfont\char200}}1
          {Й}{{\selectfont\char201}}1
          {К}{{\selectfont\char202}}1
          {Л}{{\selectfont\char203}}1
          {М}{{\selectfont\char204}}1
          {Н}{{\selectfont\char205}}1
          {О}{{\selectfont\char206}}1
          {П}{{\selectfont\char207}}1
          {Р}{{\selectfont\char208}}1
          {С}{{\selectfont\char209}}1
          {Т}{{\selectfont\char210}}1
          {У}{{\selectfont\char211}}1
          {Ф}{{\selectfont\char212}}1
          {Х}{{\selectfont\char213}}1
          {Ц}{{\selectfont\char214}}1
          {Ч}{{\selectfont\char215}}1
          {Ш}{{\selectfont\char216}}1
          {Щ}{{\selectfont\char217}}1
          {Ъ}{{\selectfont\char218}}1
          {Ы}{{\selectfont\char219}}1
          {Ь}{{\selectfont\char220}}1
          {Э}{{\selectfont\char221}}1
          {Ю}{{\selectfont\char222}}1
          {Я}{{\selectfont\char223}}1
          {é}{{\'e}}1
}
\subsection{Создание таблиц:}

\lstinputlisting[style=sql]{../src/ddl/tables.sql}

\section{Заполнить созданные таблицы тестовыми данными.}

\lstinputlisting[style=sql]{../src/dml/values.sql}

\section{Сделать скрипты для:}
\subsection{создания/удаления объектов базы данных;}
\lstinputlisting[style=sql]{../src/ddl/drop.sql}
\subsection{заполнения/удаления созданных таблиц;}
\lstinputlisting[style=sql]{../src/dml/truncate.sql}

\section{Добавить в базу данных триггеры для обеспечения комплексных ограничений
целостности.}
\lstinputlisting[style=sql]{../src/triggers.sql}
\section{Реализовать функции и процедуры на основе описания бизнес-процессов (из этапа
№1).}
\lstinputlisting[style=sql]{../src/procedures.sql}
\lstinputlisting[style=sql]{../src/functions.sql}
\section{Произвести анализ использования созданной базы данных:}
Чаще всего используются функции добавления, удаления, поиска предметов в заказе, а также сами заказы.
Все остальные таблицы создаются единожды и используются для поиска.
\section{Создать индексы и доказать, что они полезны для вашей базы данных:}
\lstinputlisting[style=sql]{../src/indexes.sql}
Индекс необходим для быстрого поиска по nationalNum, что является важной характеристикой покемона и тренеры захотят часто искать именно по этому полю.
\end{document}