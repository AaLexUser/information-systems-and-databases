\documentclass[12pt,onecolumn]{article}
\usepackage[utf8]{inputenc} % UTF8 input encoding
\usepackage[T2A]{fontenc}   % T2A font encoding for Cyrillic script
\usepackage[russian]{babel} % Russian language support
\usepackage{listings}
\usepackage{float}
\usepackage{mathtools}
\usepackage{longtable}
\usepackage{multicol}
\usepackage{lipsum}
\everymath{\displaystyle}
\usepackage{listings} 
\usepackage[usenames]{color}
\usepackage{geometry}
\usepackage{verbatim}

\geometry{
  a4paper,
  top=25mm, 
  right=15mm, 
  bottom=25mm, 
  left=15mm
}

\begin{document}
\setcounter{tocdepth}{4}
\begin{center}
    Федеральное государственное автономное образовательное учреждение высшего образования "Национальный Исследовательский Университет ИТМО"\\ 
    Мегафакультет Компьютерных Технологий и Управления\\
    Факультет Программной Инженерии и Компьютерной Техники \\
    \includegraphics[scale=0.3]{image/itmo.jpg} % нужно закинуть картинку логтипа в папку с отчетом
\end{center}
\vspace{1cm}


\begin{center}
    \textbf{Лабораторная работа №3}\\
    по дисциплине\\
    \textbf{'Информационные системы и базы данных'}\\
    \textbf{Вариант 564738291}
\end{center}

\vspace{2cm}

\begin{flushright}
  Выполнил Студент  группы P33102\\
  \textbf{Лапин Алексей Александрович}\\
  Преподаватель: \\
  \textbf{Сагайдак Алина Алексеевна}\\
\end{flushright}

\vspace{6cm}
\begin{center}
    г. Санкт-Петербург\\
    2023г.
\end{center}

\newpage
\tableofcontents
\newpage

\section{Текст задания.}
По варианту, выданному преподавателем, составить и выполнить запросы к базе данных "Учебный процесс".\\
Команда для подключения к базе данных ucheb:\\
psql -h pg -d ucheb\\
Составить запросы на языке SQL (пункты 1-7).
\begin{enumerate}
  \item 
    \begin{verbatim}
    Сделать запрос для получения атрибутов из указанных таблиц, применив фильтры по указанным условиям:
    Н_ТИПЫ_ВЕДОМОСТЕЙ, Н_ВЕДОМОСТИ.
    Вывести атрибуты: Н_ТИПЫ_ВЕДОМОСТЕЙ.НАИМЕНОВАНИЕ, Н_ВЕДОМОСТИ.ЧЛВК_ИД.
    Фильтры (AND):
    a) Н_ТИПЫ_ВЕДОМОСТЕЙ.ИД < 2.
    b) Н_ВЕДОМОСТИ.ЧЛВК_ИД = 105590.
    Вид соединения: LEFT JOIN.
  \end{verbatim}
  \item \begin{verbatim}
    Сделать запрос для получения атрибутов из указанных таблиц, применив фильтры по указанным условиям:
    Таблицы: Н_ЛЮДИ, Н_ОБУЧЕНИЯ, Н_УЧЕНИКИ.
    Вывести атрибуты: Н_ЛЮДИ.ИМЯ, Н_ОБУЧЕНИЯ.НЗК, Н_УЧЕНИКИ.ГРУППА.
    Фильтры: (AND)
    a) Н_ЛЮДИ.ИМЯ = Роман.
    b) Н_ОБУЧЕНИЯ.ЧЛВК_ИД = 112514.
    c) Н_УЧЕНИКИ.ГРУППА > 3100.
    Вид соединения: INNER JOIN.
  \end{verbatim}
  \item \begin{verbatim}
    Составить запрос, который ответит на вопрос, есть ли среди студентов группы 3102 те, кто не имеет отчества.
  \end{verbatim}
  \item \begin{verbatim}
    Найти группы, в которых в 2011 году было ровно 5 обучающихся студентов на ФКТИУ.
    Для реализации использовать подзапрос.
  \end{verbatim}
  \item \begin{verbatim}
    Выведите таблицу со средними оценками студентов группы 4100 (Номер, ФИО, Ср_оценка), у которых средняя оценка равна минимальной оценк(е|и) в группе 3100.
  \end{verbatim}
  \item \begin{verbatim}
    Получить список студентов, зачисленных ровно первого сентября 2012 года на первый курс очной или заочной формы обучения. В результат включить:
    номер группы;
    номер, фамилию, имя и отчество студента;
    номер и состояние пункта приказа;
    Для реализации использовать подзапрос с EXISTS.
  \end{verbatim}
  \item \begin{verbatim}
    Сформировать запрос для получения числа в группе No 3100 отличников.
  \end{verbatim}

\end{enumerate}
\definecolor{deepgreen}{RGB}{0,100,0}
\section{Реализация запросов на SQL.}
\lstdefinestyle{sql}{language=SQL, 
  basicstyle=\small\ttfamily,
  commentstyle=\color{deepgreen},
  stringstyle=\color{magenta}\ttfamily,
  keywordstyle=\color{blue},
  numbers=left,
  numberstyle=\scriptsize,
  numbersep=5pt,
  frame=single,
  breaklines=true,
  breakatwhitespace=true,
  showstringspaces=false,
  tabsize=4,
  inputencoding=utf8,
  extendedchars=true,
  literate={а}{{\selectfont\char224}}1
          {б}{{\selectfont\char225}}1
          {в}{{\selectfont\char226}}1
          {г}{{\selectfont\char227}}1
          {д}{{\selectfont\char228}}1
          {е}{{\selectfont\char229}}1
          {ё}{{\"e}}1
          {ж}{{\selectfont\char230}}1
          {з}{{\selectfont\char231}}1
          {и}{{\selectfont\char232}}1
          {й}{{\selectfont\char233}}1
          {к}{{\selectfont\char234}}1
          {л}{{\selectfont\char235}}1
          {м}{{\selectfont\char236}}1
          {н}{{\selectfont\char237}}1
          {о}{{\selectfont\char238}}1
          {п}{{\selectfont\char239}}1
          {р}{{\selectfont\char240}}1
          {с}{{\selectfont\char241}}1
          {т}{{\selectfont\char242}}1
          {у}{{\selectfont\char243}}1
          {ф}{{\selectfont\char244}}1
          {х}{{\selectfont\char245}}1
          {ц}{{\selectfont\char246}}1
          {ч}{{\selectfont\char247}}1
          {ш}{{\selectfont\char248}}1
          {щ}{{\selectfont\char249}}1
          {ъ}{{\selectfont\char250}}1
          {ы}{{\selectfont\char251}}1
          {ь}{{\selectfont\char252}}1
          {э}{{\selectfont\char253}}1
          {ю}{{\selectfont\char254}}1
          {я}{{\selectfont\char255}}1
          {А}{{\selectfont\char192}}1
          {Б}{{\selectfont\char193}}1
          {В}{{\selectfont\char194}}1
          {Г}{{\selectfont\char195}}1
          {Д}{{\selectfont\char196}}1
          {Е}{{\selectfont\char197}}1
          {Ё}{{\"E}}1
          {Ж}{{\selectfont\char198}}1
          {З}{{\selectfont\char199}}1
          {И}{{\selectfont\char200}}1
          {Й}{{\selectfont\char201}}1
          {К}{{\selectfont\char202}}1
          {Л}{{\selectfont\char203}}1
          {М}{{\selectfont\char204}}1
          {Н}{{\selectfont\char205}}1
          {О}{{\selectfont\char206}}1
          {П}{{\selectfont\char207}}1
          {Р}{{\selectfont\char208}}1
          {С}{{\selectfont\char209}}1
          {Т}{{\selectfont\char210}}1
          {У}{{\selectfont\char211}}1
          {Ф}{{\selectfont\char212}}1
          {Х}{{\selectfont\char213}}1
          {Ц}{{\selectfont\char214}}1
          {Ч}{{\selectfont\char215}}1
          {Ш}{{\selectfont\char216}}1
          {Щ}{{\selectfont\char217}}1
          {Ъ}{{\selectfont\char218}}1
          {Ы}{{\selectfont\char219}}1
          {Ь}{{\selectfont\char220}}1
          {Э}{{\selectfont\char221}}1
          {Ю}{{\selectfont\char222}}1
          {Я}{{\selectfont\char223}}1
}
\lstset{style=sql}
\lstinputlisting[style=sql]{../src/queries.sql}

\section{Выводы по работе.}
В ходе выполнения лабораторной работы №3 были изучены
составление запросов на языке SQL,
разные виды соединения таблиц,
подзапросы,
фильтры,
логические операторы,
выборка данных из таблиц.
\end{document}
